\documentclass[12pt]{article}

% Packages
\usepackage{graphicx}
\usepackage{enumerate}
\usepackage{tabularx}
\usepackage{array}
\newcolumntype{C}[1]{>{\centering\let\newline\\\arraybackslash\hspace{0pt}}m{#1}}
\usepackage{longtable}
\usepackage{booktabs}
\usepackage{hyperref}
\usepackage{caption}

\usepackage{placeins}
\usepackage{float}
\usepackage{multirow}
\captionsetup[table]{skip=2pt}

\usepackage{geometry}                                       
\geometry{left=2cm, right=2cm, top=2.5cm, bottom=2.5cm} 

\newcounter{TestCounter}
\setcounter{TestCounter}{0}

\newcounter{ResultCounter}
\setcounter{ResultCounter}{0}

\title{
ClinicFlow
\\\vspace{10mm}
\Large \textbf{Test Plan}
\vspace{40mm}
}
\author{ Maxim Vasiliev (400043983)
\\
Susie Yu \quad (000955758)
\\
Karl Knopf \quad (001437217)
\\
Weilin Hu \quad (001150873)
\\
Yunfeng Li \quad (001335650)
}
\date{\today}

% Document adapted from https://github.com/studouglas/GEANT4-GPU/blob/master/Documentation/TestPlan/Test%20Plan.tex 
% and http://itq.ch/pdf/Roadmap%20for%20Testing.pdf

\hypersetup{
bookmarks=true, 
colorlinks=true, 
linkcolor=black, 
citecolor=green, 
filecolor=magenta, 
urlcolor=cyan 
}


\begin{document}
\pagenumbering{gobble}
\maketitle
\newpage
\tableofcontents
\newpage
\pagenumbering{arabic}
\setlength\parindent{0pt}




\section{Introduction}


This document details our approach to testing ClinicFlow. This software will be used in a live clinic environment, so it is particularly important to ensure its robustness, reliability, and performance. We begin by testing individual components, modules and methods therein. This allows us to catch errors early in development, and help guide the design process. Next we will perform system testing, where we verify the correct operation of various user tasks. We must also perform user testing to make sure that the product is usable by clinic nurses and managers.\\

\quad


\textbf{Definitions and Acronyms} 

\vspace{-5pt}
\begin{itemize}
\setlength{\parskip}{4pt}
\setlength{\itemsep}{2pt}
\item Module: A unit or section in a clinic
\item Provider: A health care worker (A doctor, nurse,technician) 
\end{itemize}

\textbf{Tester} 

\vspace{-5pt}
\begin{itemize}
\setlength{\parskip}{4pt}
\setlength{\itemsep}{2pt}
\item Testing Team
\end{itemize}



\quad

\quad

\quad

\section{Unit Testing}

As the programs will be written in the Python 3 programming language, we can use the unittest framework. We will define test cases, with inputs and expected outputs and then be able to run this for each module in the system. We have not yet specified the names of each module and the functions contained, but we do have a general idea of what modules we would want and what they should do. 

\subsection{PatientSchedule.py} 

\textbf{Description}\\

This is the module in the program that will take in a file containing patients in the clinic, and produce a schedule for them. First we must test if scheduling system can take appropriate inputs.

\quad

\textbf{Criteria}\\

Functions create the patient schedule.

Functions output the
schedule in a useable form for the rest of the system.\\

\quad


	
	
\begin{tabular}{|c|c|c|c|}
\hline
\textbf{No.}  & \textbf{Initial State} & \textbf{Input} & \textbf{Expected Output} 
\\ \hline
1  & No dataset present & No dataset & No output 
\\ \hline
2  & No dataset present & formatted dataset & original dataset 
\\ \hline
3  & No dataset present & unformatted dataset & No output 
\\ \hline
4  & No dataset present & dataset(no repeats) & original dataset 
\\ \hline
5  & No dataset present & dataset(many repeats) & original dataset
\\ \hline
\end{tabular}\\

\quad

\quad

		
\subsection{HealthcareSchedule.py} 

\textbf{Description}\\

This is the module in the program that will take in a file containing the names and times available of the health care workers at the clinic, and it will be able to generate a schedule for them. We should first test to see if the module is able to accept the appropriate input.

\quad

\textbf{Criteria}\\

Functions create the health care schedule.

Functions output the
schedule in a useable form for the rest of the system.\\

\quad


	
	
\begin{tabular}{|c|c|c|c|}
\hline
\textbf{No.}  & \textbf{Initial State} & \textbf{Input} & \textbf{Expected Output} 
\\ \hline
1  & No dataset present & No dataset & No output 
\\ \hline
2  & No dataset present & formatted dataset & original dataset 
\\ \hline
3  & No dataset present & unformatted dataset & No output 
\\ \hline
4  & No dataset present & dataset(no repeats) & original dataset 
\\ \hline
5  & No dataset present & dataset(many repeats) & original dataset
\\ \hline
\end{tabular}\\

\quad

\quad

\subsection{Clinic.py} 

\textbf{Description}\\

Clinics can have a variety of modules such as Blood taking or x-ray. Each section can be unique or have multiples. This program module uses clinic information from a data file, which will be read into the system when a simulation is run for that clinic.

\quad

\textbf{Criteria}\\

Functions in this program module create a valid network of modules for the clinic. \\

\quad



\begin{tabular}{|c|c|c|c|}
\hline
\textbf{No.}  & \textbf{Initial State} & \textbf{Input} & \textbf{Expected Output} 
\\ \hline
1  & No dataset present & No dataset & No output 
\\ \hline
2  & No dataset present & formatted dataset & original dataset 
\\ \hline
3  & No dataset present & unformatted dataset & No output 
\\ \hline
4  & No dataset present & dataset(no repeats) & original dataset 
\\ \hline
5  & No dataset present & dataset(many repeats) & original dataset
\\ \hline
\end{tabular}\\

\quad

\quad



\subsection{SimulationEngine.py} 

\textbf{Description}\\

The system relies on the simulation engine to generate any results. This module must be able to take in the previous schedules and data, and use that to run a discrete event simulation using those schedules. We need to test both under normal conditions and under extraordinary conditions. 

\quad


\textbf{Criteria}\\

Functions in this program module can output the simulation results close to the actual 
sample dataset.\\

\quad



\begin{tabular}{|C{0.8cm}|C{2.5cm}|C{4.5cm}|C{4cm}|}
\hline
\textbf{No.}  & \textbf{Initial State} & \textbf{Input} & \textbf{Expected Output} 
\\ \hline
1  & no simulation 
data stored & Proper datasets under
normal conditions. 25
patients in simulation. & Ouputs close to sample dataset, no more than 10 minutes 
\\ \hline
2  & no simulation 
data stored & Proper datasets under
normal conditions. 50
patients in simulation. & Ouputs close to sample dataset, no more than 10 minutes 
\\ \hline
3  & no simulation 
data stored & Proper datasets under
normal conditions. Only 2
flexable workers. & Ouputs close to sample dataset, no more than 10 minutes 
\\ \hline
4  & no simulation 
data stored & Proper datasets, but all
breaks are simulataneous. & Ouputs close to sample dataset, no more than 10 minutes 
\\ \hline
5  & no simulation 
data stored & Changed datasets to
larger numbers. & Ouputs close to sample dataset, no more than 10 minutes 
\\ \hline
\end{tabular}\\

\quad

\quad

		
\subsection{Summary Output} 

\textbf{Description}\\

The final goal of this product is to generate output which aids clinic employees in decision making. For this, the output must be formatted in a way which conveys a quick and thorough understanding of simulation results and recommended schedule. There will be multiple formats, including visualizations of components as well as formats in text.

\quad


\textbf{Criteria}\\

Functions in this program module can output properly formatted text results.\\


\quad

\begin{tabular}{|c|c|c|c|}
\hline
\textbf{No.}  & \textbf{Initial State} & \textbf{Input} & \textbf{Expected Output} 
\\ \hline
1  & No previous output & Genuine data & Original output 
\\ \hline
2  & No previous output & Skewed data & No output 
\\ \hline
3  & No previous output & Invalid data & No output 
\\ \hline
4  & No previous output & Missing Data & No output 
\\ \hline
\end{tabular}\\

\quad

\quad

\subsection{User Interface}

\textbf{Description}\\

The target users of this product includes clinic managers and/or other organizations with similar scheduling goals. The user interface should be clean and intuitive as some of these users might not have extensive knowledge in software products. When the user attempts to make illegal operations,the interface should respond accordingly and direct the user towards correct behaviour.

\quad


\textbf{Criteria}\\

Functions in this program module can response to the valid user inputs.\\


\quad


\begin{tabular}{|C{0.8cm}|C{2.5cm}|C{4.5cm}|C{4cm}|}
\hline
\textbf{No.}  & \textbf{Initial State} & \textbf{Input} & \textbf{Expected Output} 
\\ \hline
1  & no previous interactions & Base case & normal response 
\\ \hline
2  & no previous interactions & Invalid interaction & No response 
\\ \hline
3  & no previous interactions & Aberrant user behaviour & No response 
\\ \hline
\end{tabular}\\

\quad


\quad
		

\subsection{Data Storage}

\textbf{Description}\\

It is important that the data is able to be saved and recalled. Detailed simulations take vital time, and users would wish to compare with previous results to evaluate the sensitivity of the system to variables. Storage errors should also not interfere with previously stored data.

\quad


\textbf{Criteria}\\

Functions in this program module can successfully save the valid dataset.\\


\quad

\begin{tabular}{|C{0.8cm}|C{2.5cm}|C{4.5cm}|C{4cm}|}
\hline
\textbf{No.}  & \textbf{Initial State} & \textbf{Input} & \textbf{Expected Output} 
\\ \hline
1  & no current data stored & Valid simulation output data & data successfully stored 
\\ \hline
2  & no current data stored & Missing data & data unsuccessfully stored 
\\ \hline
3  & no current data stored & Incorrectly formatted data input & data unsuccessfully stored 
\\ \hline
\end{tabular}\\




%------------------------------------------------------------------------------


\newpage



\section{System Testing}





\subsection{Login} 


\textbf{Description}\\

The purpose of user login is to ensure that only valid users allowed to in access the system. There are two types of account. One is the  administrator account which has full authority to manipulate the data and control all operable functions of the system. Another one is the viewer account which only can view the information.  Testing retrieves the input account and password and match with the account information in an existing database to determine whether the user is valid and what the user can do.

\quad


\textbf{Criteria}\\

Meet the expected outputs. \\


\quad


\begin{tabular}{|C{0.8cm}|C{2.5cm}|C{4.5cm}|C{4cm}|C{3cm}|C{1.5cm}|}
\hline
\textbf{No.}  & \textbf{Initial State} & \textbf{Input} & \textbf{Expected Output} 
\\ \hline
1  & Login Page Empty input
field of account and
password.
Click login. & Empty input of
one of the input
fields or both. & Stay at same
page and error
message of
empty input.
\\ \hline
2  & Login Page Empty input
field of account and
password. & Valid input of
account and
password of
administrator
account. Click
login. & Redirect to the
main page of
the application,
and give user
full authority of
control. 
\\ \hline
3  & Login Page Empty input
field of account and
password. & Valid input of
account and
password of
viewer account .
Click login. & Redirect to the
main page of the
application, user
only allowed to
view the data
and schedule. 
\\ \hline
4  & Login Page Empty input
field of account and
password. & Invalid of
account or
invalid
password. Click
login. & Stay at the
same page and
present an error
message of the
invalid login. 
\\ \hline
5  & Application main page & Click logout. & Redirect to the
login page. 
\\ \hline
\end{tabular}\\


\newpage





\subsection{Insertion and Storage of Data} 


\textbf{Description}\\

The application should allow administrator account user to insert data such as patient procedure, doctor and nurse shift hours and other necessary data into corresponding databases. The testing will compare the inserted data with the data stored in database. There exists a checking module to validate the input values.

\quad


\textbf{Criteria}\\

Meet the expected outputs. \\


\quad



\begin{tabular}{|C{0.8cm}|C{2.5cm}|C{4.5cm}|C{4cm}|C{3cm}|C{1.5cm}|}
\hline
\textbf{No.}  & \textbf{Initial State} & \textbf{Input} & \textbf{Expected Output} 
\\ \hline
1  & Application main page,
admin account. & Click Add Data. & Redirect to the
Application
adding data
page. 
\\ \hline
2  & Application adding data
page. & Add the
inexistent data
to target
database with
correct data
type. Click
submit. & Stay in same
page and all
inputs are
cleaned. Data
appear in the
corresponding
database. 
\\ \hline
3  & Application adding data
page. & Add the data to
target database
with incorrect
data type or
incorrect
pattern. Click
submit. & Stay in same
page and save
valid data.
Error messages
indicate the
invalid inputs. 
\\ \hline
4  & Application adding data
page. & Add the data
which already
existed in target
database. Click
submit. & Stay in same
page and save
data. Error
message
indicates the
redundancy of
data.
\\ \hline
5  & Application adding data
page. & Add data which
out of domain
(such as
reservation date
in past days) to
target database.
Click submit. & Stay in same
page and save
data. Error
message
indicates that
the data out of
valid range. 
\\ \hline
6  & Application main page,
viewer account. & Click Add Data. & Error message. 
\\ \hline
\end{tabular}\\






\subsection{View, Modify and Delete Data} 

\textbf{Description}\\

Administrator account user can view, modify and delete existed data in database. Viewer account user only can read the data in the database.  According to the selection of user, retrieve corresponding database and display the content on user interface. Changes and deletion created by administrator account user should be stored into database. Testing checks whether the application display right data, and  whether the changes synchronize with database. Test the validation of the database when unexpected actions happen.

\quad


\textbf{Criteria}\\

Meet the expected outputs. \\


\quad

\begin{tabular}{|C{0.8cm}|C{2.5cm}|C{4.5cm}|C{4cm}|C{3cm}|C{1.5cm}|}
\hline
\textbf{No.}  & \textbf{Initial State} & \textbf{Input} & \textbf{Expected Output} 
\\ \hline
1  & Application main page,
admin and viewer account & Click View
Data. & Redirect to the
View Data page. 
\\ \hline
2  & Application view data
page, admin and viewer
account. & Select target
database, click
view. & Stay at same
page. Display
the content of
all data from
corresponding
database. 
\\ \hline
3  & Application view data
page, admin and view
account. & Select target
database and
give specific
condition, click
view. & Stay at same
page. Display
the content of
target data from
corresponding
database. 
\\ \hline
4  & Application view data
page, admin account. & Modify the
displayed data
and change the
value by another
valid value.
Click save. & Stay at same
page. The value
in display area
and database
have been
changed. 
\\ \hline
5  & Application view data
page, admin account. & Modify the
displayed data
and change the
value by invalid
value( empty for
required value
or out of valid
range). Click
save. & Stay at same
page, no change
happen in
displayed data
or database.
Error messages
indicate the
unexpected
changes. 
\\ \hline
\end{tabular}\\

\newpage



\vspace{10pt}

\begin{tabular}{|C{0.8cm}|C{2.5cm}|C{4.5cm}|C{4cm}|C{3cm}|C{1.5cm}|}
\hline
\textbf{No.}  & \textbf{Initial State} & \textbf{Input} & \textbf{Expected Output} 
\\ \hline
6  & Application view data
page, admin account. & Delete whole
one row of data
by clicking
deletion button
at end of the
row. Click save. & Stay at same
page. The
deleted row
disappear and
the data in
database is
deleted as well. 
\\ \hline
7  & Application view data
page, admin account. & Delete whole
one row of data
if the data is
expected to use
in future
(Patient
reservation in
next few days).
Click save. & Stay at same
page. Pop up a
deletion
confirmation
window.
Confirming the
deletion will
remove the row
from the list.
Cancel deletion
will save the
data. 
\\ \hline
8  & Application view data
page, viewer account. & Modify the
displayed data. & Stay at same
page. Data is
not editable. 
\\ \hline
9  & Application view data
page, viewer account. & Click delete
button. & Stay at same
page. Deletion
button does not
exist.
\\ \hline
10  & Application view data
page, viewer account. & Click save. & Stay at same
page. Save
button does not
exist. 
\\ \hline
\end{tabular}\\


\newpage

\subsection{View Schedule and Modify Schedule}


\textbf{Description}\\

The application allows administrator account user and viewer account user to view the schedules. Administrator account user also has the permission to adjust the schedule. Testing ensures that the changed schedule doesn’t have conflicts. 

\quad


\textbf{Criteria}\\

Meet the expected outputs. \\


\quad

\begin{tabular}{|C{0.8cm}|C{2.5cm}|C{4.5cm}|C{4cm}|C{3cm}|C{1.5cm}|}
\hline
\textbf{No.}  & \textbf{Initial State} & \textbf{Input} & \textbf{Expected Output} 
\\ \hline
1  & Application main page,
admin and viewer
account. & Click view
schedule. & Redirect to the
view schedule
page. 
\\ \hline
2  & Application view schedule
page, admin and viewer
account. & Select type and
date of the
schedule. Click
view. & Stay at same
page. The page
displays the
existing
schedule. 
\\ \hline
3  & Application view schedule
page, admin account. & Click adding
button and add
a new reserved
time into blank
area of schedule.
Click save. & Stay at same
page. The new
time period is
added into
schedule.The
new schedule is
saved into
database. 
\\ \hline
4  & Application view schedule, page admin account. & Click on adding 
button on a
reserved area.
Click save & If no error
happens , stay
at same page.
No adding
button on a
reserved area. 
\\ \hline
5  & Application view schedule
page, admin account & Click on moving
button on a
reserved area
and choose an
empty area.
Click save. & If no error
happens stay at
same page. The
selected time
period is set at
new area. The
new schedule is
saved into
database. 
\\ \hline
\end{tabular}\\


\newpage



\vspace{10pt}

\begin{tabular}{|C{0.8cm}|C{2.5cm}|C{4.5cm}|C{4cm}|C{3cm}|C{1.5cm}|}
\hline
\textbf{No.}  & \textbf{Initial State} & \textbf{Input} & \textbf{Expected Output} 
\\ \hline
6  & Application view schedule
page, admin account & Click on moving
button on a
reserved area
and choose a
reserved area.
Switch the
reservation.
Click save. & If no error
happens, stay at
same page. The
selected time
period is
switched with
another one.
The new
schedule is
saved into
database. 
\\ \hline
7  & Application view schedule
page, admin account. & Application
validates the
changes of
schedule. & Checks on the
sequences of
procedures in
clinic. If adding,
moving, and
switching
violate the the
sequence, show
error message. 
\\ \hline
8  & Application view schedule
page, admin account. & Click delete
button to
remove a
reservation.
Click save. & Pop up a
window for
deletion
confirmation. If
continue to
delete, remove
the reservation.
Save the new
schedule to
database. 
\\ \hline
9  & Application view schedule
page, viewer account. & Click adding or
click moving or
click delete or
click save. & Stay at same
page. No adding
button, no
moving button,
no deletion
button, no save
button. 
\\ \hline
\end{tabular}\\


\newpage


\section{Non-Functional Requirements Testing}

\subsection{Usability}

\textbf{Description}

\vspace{-2pt}
\begin{itemize}
\setlength{\parskip}{4pt}
\setlength{\itemsep}{2pt}
\item We will list the most frequently performed tasks, and the development
		team will use the product to complete them. We count the number of
		mouse clicks.
\item We will invite five doctors and five nurses to use our product and
		give them a three minutes demonstration on how to complete a certain
		task. After that, the participants will try to complete the same task.
\item We will invite five doctors and five nurses to use our product, but
		we will not give a demonstration on how to complete a certain task.
		Instead, the participants will try to complete it based
		on their previous experience and the hints provided by the application.		
\end{itemize}




\textbf{Criteria}

\vspace{-2pt}
\begin{itemize}
\setlength{\parskip}{4pt}
\setlength{\itemsep}{2pt}
\item The average number of mouse clicks should be less than five.
\item The participants can complete the same task correctly in three minutes.
\item The participants can complete the required tasks in five minutes and
		they should encounter fewer than three errors.  
\end{itemize}

\quad


\begin{tabular}{|C{0.8cm}|C{2.5cm}|C{4.5cm}|C{4cm}|C{3cm}|C{1.5cm}|}
\hline
\textbf{No.}  & \textbf{Initial State} & \textbf{Input} & \textbf{Expected Output} 
\\ \hline
1  & Application is running
 & Performing generate simulation schedule tasks & Less than five mouse clicks 
\\ \hline
2  & Application is running & New users with instructions to complete import data task & 
Complete it in three minutes 
\\ \hline
3  & Application is running & New users without instructions to complete import data task & 
Complete it in three minutes and no more than three errors 
\\ \hline
\end{tabular}

\newpage



\subsection{Performance Testing}


\textbf{Description}

\vspace{-2pt}
\begin{itemize}
\setlength{\parskip}{4pt}
\setlength{\itemsep}{2pt}
\item The development team will use the product to complete the most frequently performed tasks.
		We will calculate the time from the start-up of the application to the completion of the work.
\item The development team will generate 100000 data points, which is ten times more
		than the expected data points. We will input those data points into our
		application. Next, we will input 100 times more data points into the application.
\item The development team will generate random data points, which contain problems such as
		incorrect formatting data and illegal characters. We will input those data points into
		our application.
\end{itemize}




\textbf{Criteria}

\vspace{-2pt}
\begin{itemize}
\setlength{\parskip}{4pt}
\setlength{\itemsep}{2pt}
\item The average time should be less than five minutes. 
\item The application does not crash, and there are no obvious latencies
		(less than 5 seconds for each task) when performing
		the common tasks in the first case. The application can crash in the second case.  
\item The application does not crash and prompts the users that the input data is not
		correct and how they could correct it.  
\end{itemize}

\quad


\begin{tabular}{|C{0.8cm}|C{2.5cm}|C{4.5cm}|C{4cm}|C{3cm}|C{1.5cm}|}
\hline
\textbf{No.}  & \textbf{Initial State} & \textbf{Input} & \textbf{Expected Output} 
\\ \hline
1  & Application is not running
 & Start application and generate simulation schedule & Less than five minutes
\\ \hline
2  & Application is running & import 100 thousand data & not crash 
\\ \hline
3  & Application is running & import 10000 thousand data & crash 
\\ \hline
\end{tabular}





\end{document}